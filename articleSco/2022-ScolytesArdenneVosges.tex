%% The '3p' and 'times' class options of elsarticle are used for Elsevier CRC
%% Add the 'procedia' option to approximate to the Word template
%\documentclass[3p,times,procedia]{elsarticle}
\documentclass[3p,times]{elsarticle}

%% The `ecrc' package must be called to make the CRC functionality available
\usepackage{ecrc}
\usepackage{color,soul}
\usepackage{multirow}
\usepackage{booktabs}
\graphicspath{{images/}}

\usepackage{lscape}


\volume{00}
\firstpage{1}
\journalname{Forest ecology and management}
\runauth{Gilles A. et al.}

%% Give the abbreviation of the Journal. %% A user-supplied logo with the name <\jid>logo.pdf will be inserted if present.
\jid{feam}
\jnltitlelogo{Forest ecology and management}
\CopyrightLine{2022}{Published by Elsevier Ltd.}

\usepackage{amssymb}
\usepackage{lineno}

\usepackage{hyperref}
\usepackage{subfig}
\usepackage[export]{adjustbox}

% if you have landscape tables
%\usepackage[figuresright]{rotating}
% add words to TeX's hyphenation exception list
%\hyphenation{author another created financial paper re-commend-ed Post-Script}

\begin{document}

\begin{frontmatter}

%% Title, authors and addresses

%% use the tnoteref command within \title for footnotes;
%% use the tnotetext command for the associated footnote;
%% use the fnref command within \author or \address for footnotes;
%% use the fntext command for the associated footnote;
%% use the corref command within \author for corresponding author footnotes;
%% use the cortext command for the associated footnote;
%% use the ead command for the email address,
%% and the form \ead[url] for the home page:
%%
%% \title{Title\tnoteref{label1}}
%% \tnotetext[label1]{}
%% \author[label1,label2]{<author name>}
\author[label2]{Gilles Arthur}
\ead{arthur.gilles@uliege.be}
\author[label2]{Lisein Jonathan}
%\ead{liseinjon@hotmail.com}
\author[label2]{Claessens Hugues}

%% \ead[url]{home page}
\fntext[label2]{Liège University - Faculty of Gembloux Agro-Bio Tech - unit of forest ressources managment}
%\fntext[label1]{Forêt.Nature asbl}
%\cortext[cor1]{}
%% \address{Address\fnref{label3}}
%% \fntext[label3]{}

\dochead{Original research papers}
%% Use \dochead if there is an article header, e.g. \dochead{Short communication}
%% \dochead can also be used to include a conference title, if directed by the editors
%% e.g. \dochead{17th International Conference on Dynamical Processes in Excited States of Solids}

%pdftotext 2021-phytoSpy.pdf - | tr -d '.' | wc -w pour le nombre de mot

\title{Bark beetle infection of spruce differs between Belgium and north France : a remote sensing analysis of 2016-2021 dieback}

\begin{abstract}

\end{abstract}

\begin{keyword}
% nutrient regime \sep moisture regime
\end{keyword}

\end{frontmatter}

\linenumbers

\pagebreak
\section{Introduction}

\begin{itemize}
	\item Aire de répartition de l'épicéa
	\item Scolyte description générale, plus précision typographe chalcographe
	\item Evolution des dégats lié au scolyte dans le monde
	\item Début de la crise en Wallonie + Vosges en 2018
%	\item Objectif de l'article: caractérisation des attaque de scolytes en Wallonie et dans les Vosges selon deux     variables environnementales +paramètres macro 
\end{itemize}

\section{Material and methods}
\subsection{Study area}
The study area was located in the south of Belgium and in the north east of France. We study 2 regions: Wallonia and Grand-Est.
This two regions are covered by 7 Sentinel-2 tiles (6 tiles for the Wallonia and 1 for the Vosges).

In Wallonia, the altitude varies between 100 and 700m. The walloon forest covers 554 600 Ha. The norway spruce stand occuped 139 600 Ha \citep{Alderweireld_2015}. For this study, we selected only spruce trees over 15 m and we have worked on %nombre Ha total 
Ha. Two thirds of the Walloon spruce forest is located above 400m altitude
The Wallon climate is continental(\citep{Metzger_2005}. %integrer figure climat
Over the 1989-2020 period, the average temperature was 9,7°C et the average sum of rainfall was 1118mm. During 2018, the average temperature was 9.6°C et the average sum of rainfall was 919 mm (data of Institut Royal Métérologique). (fig %%%)
%The month of july 2018 is considered as a arid month. 

In the Grand-Est, the elevation is between 100m and 1300m. The Grand-est forest occupies 1 939 000 ha. The norway spruce forest covers 136 000 Ha (\citep{francais_donnees_2022}).
n this study, we worked on 125,000 ha of spruce in this region.

%nous en detecte 125 000 ha
The majority of norway spruce stand of this region grow between 400m and 900m.
During the 2012-2020 period, the average temperature was 9,6°C et the average sum of rainfall was 1347mm. During the year 2018, the average temperature was 10,2°C et the average sum of rainfall was 1233 mm.


%\subsection{Description zone de la Zone d'étude}

%\begin{itemize}
%	\item Description zone d'étude générale + tuile S2 traitées (figure: %\ref{fig:rep_vosg}) 
%	\item Description générale de la forêt wallonne 
%	\item Description de la pessière wallonne 
%	\item Description générale de la foret vosgienne 
%	\item Description de la pessière vosgienne
	%\item Comparaison Température et précipitation vosges et Wallonie entre moyenne trentenaire et données pour l'année 2018 (figure reffig:diagOT)
	
%\end{itemize}



\subsection{DEM and Slope orientation}


We have used the digital surface model (DEM) data from the Copernicus Land Monitoring Service \citep{DEM_copernicus}  at a resolution of 25mX25m for all elevation data and slope calculations.

 %European Union, Copernicus Land Monitoring Service <year>, European Environment Agency (EEA)
%\begin{itemize}
%	\item Provenance des données de MNT
%	\item Méthodologie de calcul des sous-secteur radiatif.
%\end{itemize}

Solar orientation influences bark beetle capture in pheromone traps (\citep{AFR64}). We determined this solar orientation using the Delvaux and Galoux definition of the 3 radiative sub-sectors  (\citep{Delvaux_galoux}):
%For the three sub-sectors (cold slope, plateau, hot slope), we applied the definition of Delvaux and Galoux (1962):
\begin{itemize} 
\item Plateau: plateau and low slope (slope less than 12° or 20%) that does not create a particular microclimate.
\item cold slope: slope greater than 12° or 20% facing north and valley bottom. These are shady, cool and humid areas.
\item warm slope: slope greater than 12° or 20% facing south. In this sub-sector the air is warmer and drier and the temperature difference between day and night is greater.
\end{itemize}
Based on this definition and on the DEM, we produced radiative sub-sector maps for Wallonia and the Grand-Est.

\subsection{Mapping of spruce dieback and mortality by analysis of sentinel-2 time-serie}


The European Union’s earth observation programme, with its satellite twin constellation Sentinel-2A and Sentinel-2B, provides free earth imagery with a high revisit time. 
Sentinel-2 (S2) satellites carry multispectral sensor with a ground resolution up to 10 m. 
S2 imagery have been intensively used recently for forestry purpose, including for the monitoring of bark beetle outbreaks. 
Low and Koukal \citep{low_phenology_2020} have modelled phenology courses of vegetation indices to detect forest disturbances. 
They have properly mapped Bark beetle infestation in Austrian spruce stands.
Ali \textit{et al.} \citep{ali_canopy_2021} have used multi-years time series remote sensing data in order to detect early bark beetle infestation in Germany. 
They have highlighted the potential of S2 data for the production of reliable infestation maps.
Barta \textit{et al.} \citep{barta_early_2021} have studied spectral trajectories of nine bands and six vegetation indices from S2 imagery for the 2018 vegetation season. 
They have confirmed the superiority of multi-date data for the classification by Random Forest of infested stands in the Czech Republic.

In this present research, the detection of bark beetle infestation is realized by using dense time series of S2 imagery following the methodology developped by Dutrieux \textit{et al.} \citep{dutrieux_package_2021}.
Vegetation changes are tracked by means of a phenology metric, the \textit{SWIR Continuum Removal} ($SWIR_{CR}$) indice.
All S2 acquisitions are used in the analyses, provided that the cloud couver do not excess 35 percent.  Bottom Of Atmosphere reflectance images (L2A product) are downloaded from the Theia data cluster \citep{theia_team_value-added_nodate} for all the 6 granules, which are tiles of 100km x 100km, that covers Wallonia. 
For north France, one single granule covers the Vosges mountains.
The $SWIR_{CR}$ is based on three spectral bands, the near-infrared, the shortwave infrared 1 band and the shortwave infrared 2, and is sensitive to the foliage water content (figure \ref{fig:harmo}).
Seasonal variation of $SWIR_{CR}$ for healthy stand is modelled and a bark beetle attack is detected if the observations deviates from the healthy phenology trajectory. 
Figure \ref{fig:harmo} illustrates a time-serie of $SWIR_{CR}$ observations (grey dots) for one pixel. 
In 2018, the observations goes beyond the threshold represented by the purple-dashed line, which shows that the spruce stand suffer from a serious stress induced by a bark beetle attack.
A bark beetle outbreak is confirmed as soon as $SWIR_{CR}$ vegetation indice show a stress for at least three consecutive times.
In parallel to the detection of bark beetle stress, stand cutting and thinning are subject of particular attention. 
Bare soil is detected by using a combination of red, green and shortwave infrared reflectance values.
Cutting are thus taken into account and are classified either as normal harvest cutting or as sanitary thinning based on the health status prior to the cutting.
The analysis of image time-serie is thus quite straightforward and is performed individually pixel per pixel starting from the 2016 year, which is the beginning of S2 acquisitions. 
The dense time-serie covers the 2016-2021 period and count a minimum of 180 acquistion dates. 
The health status is summarized in annual health maps by means of four classes ; healthy, bark beetle attached, cutted and sanitary thinning.

\begin{figure}
	\centering
	\includegraphics[width=\textwidth]{fctHarmo.png}
	\caption{Bark beetle infestation map are computed by detecting change in the $SWIR_{CR}$ phenology metric. The \textit{SWIR Continuum Removal} is computed using three bands from Sentinel-2 imagery for every single acquisition date and his value is compared to a threshold (purple dashed line) in order to detect vegetation stress. If a stress is detected three consecutive times, we assume that a bark beetle infection occured.}
	\label{fig:harmo}
\end{figure}

Our approach of bark beetle detection is only suitable for spruce, as it is closely related to the phenological course of healthy spruce forest.
An essential prerequisite is thus to have a proper mapping of spruce stands.
For the south of Belgium, we use existing reliable composition maps \citep{bolyn_forest_2018} computed from remote sensing data in order to restrict our analysis to spruces.
In Vosges mountains, the composition map comes from the French Mapping agency (Forest BD version 2). 
Composition of forest stand is determined by photointerpretation and forest stands identifyed as "spruce or fir" serve as starting point to limit our analysis.
Time series are a convenient means to track phenology changes. 
More broadly than the dectection of bark beetle infestion, phenology courses are highly suitable for forest tree species discrimination \citep{lisein_discrimination_2015,grabska_forest_2019,ma_tree_2021}.
We have used S2 spectral bands courses along the vegetation season to refine the determination of species present in the area interpreted as "spruce or fir" in Vosges.
The objective is to identify and remove every area that are not spruce stand, as pixels located on others species than spruce are likely to be wrongly detected as a bark beetle attack.
All S2 spectral bands were first summarized for each of the four trimesters of the year, by simply averaging all observations occuring during the trimester.
Then, a Random Forest algorithm was trained on these synthetic intra-annual time serie to discriminate spruce from non-spruce pixels, based on a training set of observation from Belgium \citep{bolyn_forest_2018}.
Eventually, this Random Forest classifier was applied on "spruce and fir" area of Vosges and bark beetle detection was carried on only for pixels detected as spruce. 






\subsection{Data analysis}

%Depuis 2018, des attaques massives de scolytes tuant les épicéas frappent la Wallonie. Suite à ces évenements,les forestiers se sont interrogés sur cerains facteurs topographiques semblant avoir fortement influencé les attaques de scolytes.
%Les pessières situées en basse altitude semblent avoir été plus touché ainsi que les peuplement situé sur des versants sud.

%Pour caracteriser les attaques de scolytes, nous avons appliquer la méthode des random forest afin de selectionner les 2 facteurs topographiques influençant le plus les attaques de scolytes. CEs deux facteur sont l'altitude et les sous-radiatif. Nous avons ventilé l'altitude par classe de 100m et conservé les trois classes de sous secteurs définis par delvaux et galoux.

%Ensuite, afin de determiner les classes de ces facteurs les plus impactés par le scolyte, nous avons estimé les surfaces scolytés pour chacune des classes de chaque facteurs sur base des cartes d'état sanitaire pour chaque année de la période 2016-2021.

%La carte d'etat sanitaire de la pessière a été subdivisée en tuile de 50*50m (25 pixel de 10X10m) comprenant au minimum 17 pixel de 10mX10m d'épicéas. Une tuile est considérée comme %scolytée quand minimum 3 pixels sur 25 sont scolytés.

%Pour chaque tuile, la classe d'altitude et le sous-secteur ont été extraits.

%Nous avons calculé le ratio du nombre tuiles scolyté d'une classe divisé par le nombre total de tuiles de la classe (probabilité de présence) pour chacune des classes d'altitude et de sous-secteurs.


%\begin{align*}

%$presence\,of\,probability = \frac{tiles\, affected\, by\, the\, bark\, beetle\, of\, a\, class}{total\, number\, of\, tiles\, in\, the\, class}$

%\end{align*}


Since 2018, massive bark beetle attacks killing spruce trees have been occurring in Wallonia. Following these events, foresters have asked themselves about certain topographical factors that seem to have strongly influenced bark beetle attacks.
The spruce forests located at low altitude seem to have been more affected as well as the stands located on southern slopes.

To characterise the bark beetle attacks, we applied the random forest method to select the two topographic factors that most influenced the bark beetle attacks. These two factors are altitude and sub-radiation. We broke down the altitude by 100m classes and kept the three sub-sector classes defined by Delvaux and Galoux (1972).

Then, in order to determine the classes of these factors most impacted by the bark beetle, we estimated the bark beetle areas for each class of each factor based on the health status maps for each year of the period 2016-2021.

The spruce health map was subdivided into 50*50m tiles (25 pixels of 10X10m) comprising at least 17 pixels of 10mX10m spruce. A tile is considered to be barked when at least 3 out of 25 pixels are barked.

For each tile, elevation class, sub-sector and the sanitary status were extracted.

We calculated the ratio of the number of tiles attacked by bark beetle in a class divided by the total number of tiles in the class (probability of occurrence) for each of the elevation and sub-sector classes.




\begin{itemize}
	\item Random forest  
	

	\item test de student
	\item the probability of presence of bark beetles
The probability of bark beetle presence is the area affected by bark beetle attacks on the total spruce area. 



%Equation 
\end{itemize}

\section{Results}

\subsection{ Elevation vs bark beetle presence}
The variation of the probability of presence of bark beetles for Wallonia and Grand-Est for the period 2017-2021 is described in the figure 
\ref{alti_sco}.
 The altitude has been subdivided into the same 12 elevation classes for both regions. The graphs corresponding to the variation of the probability of presence in Wallonia are in the upper part of the figure and those for the Grand-Est in the lower part.
\begin{landscape}
\begin{figure}
\centering
	\includegraphics[width=\textwidth]{graphe_wall_GDE.png}
     \caption{XXXXAPragrapheprecedent.}
	\label{alti_sco}
\end{figure}

 \end{landscape}

For the Walloon spruce stand, there is an increase of the probability de presence of bark beetle for all altitude classes until 2020.
 In 2021, the probabilty of bark beetle presence decrease for all altitude classes.
  
This figure also shows that the bark beetle crisis started in 2018. Indeed, during this year a strong increase in the 100-200m and 200-300m altitude classes is observed. These two altitude classes are more affected during this crisis.  
The decrease the probabilty of bark beetle presence follow an altitudinal gradient. The higher the norway spruce stand grows at an altitude, the lower the the probabilty of bark beetle presence.


For the Grand-Est region, there is an increase in the probability of bark beetle presence between 2019 and 2021. Unlike the Walloon spruce forest, there is no clear altitudinal gradient in the Vosges.  
As in Wallonia, the 200-300m altitude class is strongly affected by the bark beetle. The probability of presence decreases along an altitudinal gradient between the altitude classes 200-300 and 400-500m. However, above 500m the probability of bark beetle increases up to the altitude class 700-800m.





\begin{itemize}
	\item Description figure \ref{fig:sco_alti} 
	\item augmentation de la probabilité de présence jusqu'en 2020 et diminution en 2021
	\item Wallonie: Diminution de la probabilité de présence de scolyte avec l'augmentation de l'altitude 
	\item Vosges pas de relation clair avec l'altitude. Cependant, les classes d'altitude 2, 11 et 12 semblent + touchées que les autres classes d'altitude
	
	\item Wallonie + Vosges: Augmentation de la probabilité de présence de scolyte avec le temps quelque soit la classe d'altitude.
\end{itemize}



\subsection{Sous-secteur radiatif vs probabilité de présence de scolyte}


\begin{figure}
\centering
	\includegraphics[width=\textwidth]{evol_ss_GDE_wall.png}
     \caption{XXXXAPragrapheprecedent.}
	\label{ss_sco}
\end{figure}

%\begin{itemize}
%	\item Description figure \ref{fig:ss_wall} 
%	\item Wallonie: Différences significative entre les différents sous secteurs. Les sous secteur froid sont  plus touchés que les sous secteur chaud et les plateaux. Les plateaux sont moins touché en Wallonie.
%	\item Vosges: pas de différence significative entre les sous secteurs ( \ref{fig:ss_vosg})
	
%\end{itemize}

%sous-secteur
%graphe + descruipton




\section{Discussion}

\subsection{Différence entre Vosges et Wallonie}
\begin{itemize}
	\item Différence climat (Climat semicontinental/montagnard vs climat tempéré océanique)
	\item Différence sylvicole ( Wallonie futaie régulière exploitable vs Vosges peuplement + mélangé et moins exploitable en haute altitude)
	\item Sommet des vosges epicéas endémiques vs épicéas en plantations (résilience peuplement )
	\item adaptation ep à condition plus rude en versant sud que nord 
	\item meilleur surveillance des forestiers sur versant sud que nord 
	
\end{itemize}

\subsection{Facteur déterminant l'attaque par l'épicéa ou le scolyte}

\begin{itemize}
	\item Discussion généralisation de modèle scolyte/ dépérissement des épicéas
	\item est ce la Biologie du scolyte/ ou le stress de l'épicéa qui conditionne le dépérissement massif ?
	
\end{itemize}

\section{Conclusion}

Dépérissement différents pour les Vosges et la Wallonie.

\section{Figure}

\begin{figure} [htbp] 
	\centering
	\includegraphics[width=1\textwidth]{waql.png}
	\caption{Zones d'études avec le MNT (XXX) et les tuiles du satellite Sentinelle 2 employées(XXX légende carré rouge).}
	\label{fig:situ}
\end{figure}


\begin{figure} [htbp] 
	\centering
	\includegraphics[width=1\textwidth]{diagrammeOT-page001.png}
	\caption{Walter and Lieth climatic diagram comparison for Ardenne (up) and Vosges (down). Left diagram show the average recent climate, and rigth one illustrates the year of 2018.}
	\label{fig:diagOT}
\end{figure}


\begin{figure} [htbp] 
		\centering
		\includegraphics[width=1\textwidth]{Wall_vs_vosges.png}
		\caption{Probabilité de présence de scolyte en fonction de l'altitude pour la Wallonie et les Vosges}
		\label{fig:sco_alti}
\end{figure}
	

%\begin{figure}[htbp]
%	\begin{minipage}[b]{1 \linewidth}
%		\centering
	%	\includegraphics[width=1\textwidth]{evol_ss_wal.png}
%		\caption{Évolution de la crise du typographe en région wallonne en fonction des sous-secteurs.}
%		\label{fig:ss_wall}
		%\caption sert à insérer une légende
%	\end{minipage}\hfill
%	\vspace{1cm}
%	\begin{minipage}[b]{1 \linewidth}
%		\centering
%		\includegraphics[width=1\textwidth]{evol_ss_vosges.png}
%		\caption{Évolution de la crise du typographe dans les Vosges en fonction des sous-secteurs .}
%		\label{fig:ss_vosg}
%	\end{minipage}
\%end{figure}

\section{Acknowledgements}

This research has been funded thanks to the \textit{RegioWood II} project.

\bibliographystyle{elsarticle-num}
%\bibliographystyle{elsarticle-harv}\biboptions{authoryear}
\bibliography{Scolyte.bib}
\end{document}