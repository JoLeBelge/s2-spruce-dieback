\section{Références}


\noindent DUTRIEUX, R., FERET, J.-B., OSE, K., and DE BOISSIEU, F., 2021.
Package Fordead. \url{https://doi.org/10.15454/4TEO6H} \vspace{\baselineskip}

\noindent Dutrieux, R., Feret, J.-B., and Ose, K., July 2021. Mise au
point d’une méthode reproductible pour le suivi généralisé des
dégats de scolytes par télédétection satellitaire. ONF Rendez-vous
techniques, (69-70):37–44.  \url{https://www.onf.fr/onf/+/cec::les-rendez-vous-techniques-de-lonf-no69-70.html} \vspace{\baselineskip}


\noindent Petit S., Cordier S., Claessens H., Ponette Q., Vincke C., Marchal D., Weissen F. (2017). Fichier écologique des essences. Forêt.Nature, UCLouvain-ELIe, ULiège-GxABT, SPWARNE-DNF. fichierecologique.be, 22/12/2021 \vspace{\baselineskip}

\noindent Centre National de la propriété forestière, Decouvrez BioClimSol. \url{https://www.cnpf.fr/n/decouvrez-bioclimsol/n:4199} \vspace{\baselineskip}

\noindent Lisein J., 2022, Guide méthodologique: Analyse des séries temporelles d'image Sentinel-2 pour la détection des épicéas scolytés. en ligne sur \url{ https://forestimator.gembloux.ulg.ac.be/pdf/methodoAnalyseSentinel2TimeSerie2021.pdf}\vspace{\baselineskip}

\noindent Van den Perre R., Bythell S., Bogaert P., Claessens H.,Ridremont F., Tricot C., Vincke C., Ponette Q.(2015).La carte bioclimatique de Wallonie : un nouveau découpage écologique du territoire pour le choix des essences forestières.FORÊT.NATURE n°135
AVRIL-MAI-JUIN: 47-58.\vspace{\baselineskip}

\noindent Wampach F., Lisein J,Cordier S.,Ridremont F., Claessens H (2017), Cartographie de la disponibilité en eau et en éléments nutritifs des stations forestières de Wallonie.FORÊT.NATURE n°143 48 AVRIL-MAI-JUIN: 48-60.\vspace{\baselineskip}